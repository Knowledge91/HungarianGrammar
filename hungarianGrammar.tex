\documentclass{article}

% hungarian 
\usepackage[magyar]{babel}
\usepackage{fontspec}

% tabular (\toprule, \midrule, \bottomrule)
\usepackage{booktabs}
% full width tabular
\usepackage{tabularx}


\begin{document}

\section{Vowel characterisation}

\begin{tabularx}{\textwidth} { X c } \toprule
	Back vowels: & a, á, o, ó, u, ú  \\
	Front unrounded vowels: & e, é, i, í­ \\
	Front rounded vowels: & ö, ő, ü, ű \\ \bottomrule
\end{tabularx}
		


\section{Plural of nouns}
Add \textbf{k} to end of noun.  \\
When last letter is \textbf{a} or \textbf{e} they change to \textbf{á}, \textbf{é} (egyetemista: egyetemisták). \\
\begin{tabularx} {\textwidth}{ X X l } \toprule
	Back vowel word & Front vowel word & \\ \midrule
	 $\downarrow$ & Last vowel unrounded & Last vowel rounded \\
	\textbf{o} & \textbf{e} & \textbf{ö} \\ \midrule
	tanár: tanérok & művész: művészek & ismerős: ismerősök \\ \bottomrule
\end{tabularx}
Exceptions: \textbf{férfi}: \textbf{férfiak}, \textbf{könyv}: \textbf{könyvek}, \textbf{toll}: \textbf{tollak}

\section{Plural of adjectives}



\end{document}