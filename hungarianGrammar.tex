\documentclass{article}

% hungarian 
\usepackage[magyar]{babel}
\usepackage{fontspec}

% tabular (\toprule, \midrule, \bottomrule)
\usepackage{booktabs}
% full width tabular
\usepackage{tabularx}


\begin{document}

\section{Vowel characterisation}
\begin{tabularx}{\textwidth} { X c } \toprule
	Back vowels: & a, á, o, ó, u, ú  \\
	Front unrounded vowels: & e, é, i, í­ \\
	Front rounded vowels: & ö, ő, ü, ű \\ \bottomrule
\end{tabularx}
Depending on the count of back, front unrounded and front rounded vowels in a word we call words with:
\begin{itemize}
\item mainly back vowels \textbf{back vowel word},
\item mainly front unrounded vowels \textbf{front unrounded word},
\item mainly front rounded vowels \textbf{front rounded word}.
\end{itemize}
		


\section{Plural of nouns}
Add \textbf{k} to end of noun.  \\
When last letter is \textbf{a} or \textbf{e} they change to \textbf{á}, \textbf{é} (egyetemista: egyetemisták). \\
\begin{tabularx} {\textwidth}{ X X l } \toprule
	Back vowel word & Front vowel word & \\ \midrule
	 $\downarrow$ & Last vowel unrounded & Last vowel rounded \\
	\textbf{o} & \textbf{e} & \textbf{ö} \\ \midrule
	tanár: tanérok & művész: művészek & ismerős: ismerősök \\ \bottomrule
\end{tabularx}
Exceptions: \textbf{férfi}: \textbf{férfiak}, \textbf{könyv}: \textbf{könyvek}, \textbf{toll}: \textbf{tollak}

\section{Plural of adjectives}
Plurals of adjectives are only formed if the adjective presents the entire predicate (e.g. A füzetek újak).
\begin{itemize}
\item Attach \textbf{-k}
\item Ending vowels \textbf{-a, -e} transform to \textbf{-á, -é} (e.g. drága - drágák)
\item if required\footnote{Word ends in consonant or i/ú/ű/ó/ő} add linking vowel \\
\begin{tabularx} {\textwidth}{ X X } \toprule
	Back vowel word & Front vowel word \\ 
	\textbf{a} & \textbf{e} \\ \midrule
	magas: magasak & egyszerű: egyszerűek \\ \bottomrule
\end{tabularx}
\end{itemize}





\end{document}